\chapterOrPart{Nomenclature}\label{sec:nomenclature}

It is good practice to include a list of used abbreviations and symbols in your report or thesis in a so-called nomenclature. Below you will find an example. Inspect the source file to see how these tables are made. In your own copy of this template, you can add, remove or alter the entries to meet your needs.
\vspace{1cm}

\def\kolomA{1.5cm} \def\kolomB{7cm} \def\kolomC{5.5cm}

% Acronyms
\begin{supertabular}{p{\kolomA}p{\kolomB}}
\multicolumn{2}{l}{\textbf{Acronyms}} \\
\\ 
\acronym{APU}{Auxiliary Power Unit}{hulpaggregaat}
\acronym{BAT}{Best Available Technology}{best beschikbare techniek}
\acronym{BDC}{Bottom Dead Centre}{onderste dode punt}
\acronym{CC}{Combined Cycle}{stoom- en gascentrale}
\acronym{CHP}{Combined Heat and Power}{warmte-krachtkoppeling}
\acronym{COP}{Coefficient of Performance}{prestatiefactor}
\acronym{CS}{Control Surface}{controle-oppervlak}
\acronym{CV}{Controle Volume}{controlevolume}
\acronym{FOD}{Foreign Object Damage}{schade wegens vreemd object}
\acronym{GWP}{Greenhouse Warming Potential}{aardopwarmingsvermogen}
\acronym{HE}{Heat Engine}{warmtemotor}
\acronym{HHV}{Higher Heating Value}{hogere verbrandingswaarde}
\acronym{HX}{Heat Exchanger}{warmtewisselaar}
\acronym{IGV}{Inlet Guide Vanes}{geleidingsschoepen aan inlaat}
\acronym{ISA}{International Standard Atmosphere}{internationale standaard atmosfeer}
\acronym{LHV}{Lower Heating Value}{lagere verbrandingswaarde}
\acronym{ODP}{Ozone Depletion Potential}{ozonafbrekend vermogen}
\acronym{OEI}{One Engine Inoperative}{één motor buiten werking}
\acronym{RH}{Relative Humidity}{relatieve vochtigheid}
\acronym{RPM}{Rounds Per Minute}{omwentelingen per minuut}
\acronym{TDC}{Top Dead Centre}{bovenste dode punt}
\acronym{TIT}{Turbine Inlet Temperature}{inlaattemperatuur turbine}
\\[+4ex]
\end{supertabular}

% Dimensionless numbers
    \begin{supertabular}{p{\kolomA}p{\kolomB}l} 
    \multicolumn{2}{l}{\textbf{Dimensionless numbers}} \\[+2ex]
    \dimensionlessnumber{Ma}{Mach number}{\up{Ma}=\vel/\vel_\up{sound}}
    \dimensionlessnumber{Nu}{Nusselt number}{\up{Nu}=h L/\lambda} % convection/conduction in fluid
    \dimensionlessnumber{Pr}{Prandl number}{\mathrm{Pr}=\nu/\alpha=\mu c_p/\lambda}
    \dimensionlessnumber{Re}{Reynolds number}{\mathrm{Re}=\rho \vel L/\mu}
    \\[+4ex]
    \end{supertabular}

% Greek symbols
\begin{supertabular}{p{\kolomA}p{\kolomB}l}
	\multicolumn{2}{l}{\textbf{Greek symbols}} \\[+2ex]
	\nomenclature{\alpha}{thermische diffusiviteit}{thermal diffusivity}{m^2/s}
	\nomenclature{\beta}{massatransportcoëfficiënt}{mass transfer coefficient}{m/s}
	\nomenclature{\delta}{infinitesimale hoeveelheid}{infinitesimal quantity}{ }
	\nomenclature{\epsilon}{emissiviteit}{emissivity}{-}
	\nomenclature{\epsilon}{rek}{strain}{-}
	\nomenclature{\eta}{efficiëntie}{efficiency}{\%}
	\nomenclature{\gamma}{warmtecapaciteit ratio}{heat capacity ratio}{-}
	\nomenclature{\lambda}{thermische geleidingscoëfficiënt}{thermal conduction coefficient}{W/mK}
	\nomenclature{\mu}{dynamische viscositeit}{dynamic viscosity}{kg/ms}
	\nomenclature{\pi}{drukverhouding}{pressure ratio}{-}
	\nomenclature{\nu}{kinematische viscositeit}{kinematic viscosity}{m^2/s}
	\nomenclature{\rho}{massadichtheid}{mass density}{kg/m^3}
	\nomenclature{\mathcal{V}}{volume}{volume}{m^3}
	\nomenclature{\omega}{ruimtehoek}{solid angle}{sr}
\\[+4ex]
\end{supertabular}

% Roman symbols
\begin{supertabular}{p{\kolomA}p{\kolomB}l}
\multicolumn{2}{l}{\textbf{Roman symbols}} \\[+2ex]
\nomenclature{A}{oppervlak}{area}{m^2}
\nomenclature{c}{warmtecapaciteit}{specific heat capacity}{J/kgK}
\nomenclature{D}{diameter}{diameter}{m}
\nomenclature{D}{massadiffusiecoëfficiënt}{mass diffusion coefficient}{m^2/s}
\nomenclature{D}{thermische dilatatiecoëfficiënt}{thermal dilation coefficient}{-}
\nomenclature{E}{totale energie}{total energy}{J}
\nomenclature{E}{modulus van Young}{Young's modulus}{N/m^2}
\nomenclature{F}{kracht}{force}{N}
\nomenclature{g}{valversnelling}{acceleration of gravity}{m/s^2}
\nomenclature{H}{enthalpie}{enthalpy}{J}
\nomenclature{\dot{H}}{enthalpiedebiet}{enthalpy flow rate}{J/s}
\nomenclature{h}{specifieke enthalpie}{specific enthalpy}{J/kg}
\nomenclature{h}{convectiecoëfficiënt}{convection coefficient}{W/m^2K}
\nomenclature{I}{elektrische stroom}{electric current}{A}
\nomenclature{L}{lengte}{length}{m}
\nomenclature{M}{moleculaire massa}{molecular mass}{kg/mol}
\nomenclature{m}{massa}{mass}{kg}
\nomenclature{\dot{m}}{massadebiet}{mass flow rate}{kg/s}
\nomenclature{n}{aantal mol}{number of moles}{mol}
\nomenclature{N}{toerental}{rotational speed}{rpm}
\nomenclature{\pres}{druk}{pressure}{Pa}
\nomenclature{Q}{warmte}{heat}{J}
\nomenclature{\qdot}{thermisch vermogen}{thermal power}{W}
\nomenclature{q}{specifieke warmte}{specific heat}{J/kg}
\nomenclature{R}{elektrische weerstand}{electric resistance}{ohm}
\nomenclature{R_\up{u}}{universele gasconstante}{universal gasconstant}{J/molK}
\nomenclature{R_\up{s}}{specifiek gasconstante}{specific gasconstant}{J/kgK}
\nomenclature{r}{compressieverhouding}{compression ratio}{-}
\nomenclature{S}{entropie}{entropy}{J/K}
\nomenclature{s}{specifieke entropie}{specific entropy}{J/kgK}
\nomenclature{T}{koppel}{torque}{Nm} % better to use M because T already used for temperature
\nomenclature{T}{stuwkracht}{thrust}{N}
\nomenclature{T}{temperatuur}{temperature}{K}
\nomenclature{U}{elektrische spanning}{voltage}{volt}
\nomenclature{U}{inwendige energie}{internal energy}{J}
\nomenclature{\vol}{volume}{volume}{m^3}
\nomenclature{\vel}{snelheid}{velocity}{m/s}
\nomenclature{\specvol}{specifiek volume}{specific volume}{m^3/kg}
\nomenclature{W}{arbeid}{work}{J}
\nomenclature{\wdot}{mechanisch vermogen}{mechanical power}{W}
\nomenclature{w}{specifieke arbeid}{specific work}{J/kg}
\nomenclature{x}{dampkwaliteit of dampfractie}{vapour fraction}{-}
\nomenclature{Y}{molfractie}{mole fraction}{-} %
\nomenclature{z}{hoogte}{height}{m}
\nomenclature{Z}{compressibiliteitfactor}{compressibility factor}{-}
\\[+4ex]
\end{supertabular}

% Superscripts
\begin{supertabular}{p{\kolomA}p{\kolomB}l}
	\multicolumn{2}{l}{\textbf{Superscripts}} \\[+2ex]
	\nomenclature{\prime}{per strekkende meter}{per meter}{1/m}
	\nomenclature{\dot{}}{debiet}{flow rate}{s^{-1}}
	\nomenclature{\bar{}}{per mol}{per mole}{mol^{-1}}
	\nomenclaturesubscript{*}{kritisch}{critical}
	\nomenclaturesubscript{isen}{isentroop}{isentropic}
	\nomenclature{o}{bij \SI{25}{\degreeCelsius}}{at \SI{25}{\degreeCelsius}}{ }
	\\[+4ex]
\end{supertabular}

% Subscripts
\begin{supertabular}{p{\kolomA}p{\kolomB}}
\multicolumn{2}{l}{\textbf{Subscripts}} \\[+2ex]
\nomenclaturesubscript{a(tm)}{atmosferisch}{atmospheric} 
\nomenclaturesubscript{a}{lucht}{air}
\nomenclaturesubscript{bb}{zwart lichaam}{blackbody}
\nomenclaturesubscript{cr}{kritisch punt}{critical point}
\nomenclaturesubscript{elek}{elektrisch}{electric}
\nomenclaturesubscript{f}{brandstof}{fuel}
\nomenclaturesubscript{g}{generatie}{generated}
\nomenclaturesubscript{i}{isentroop}{isentropic}
\nomenclaturesubscript{0}{dode toestand}{dead state}
\nomenclaturesubscript{\pres}{bij constante druk}{at constant pressure}
\nomenclaturesubscript{p(rim)}{primair}{primary}
\nomenclaturesubscript{R}{gereduceerd}{reduced} 
\nomenclaturesubscript{s}{specifiek}{specific}
\nomenclaturesubscript{s(ec)}{secundair}{secundary}
\nomenclaturesubscript{st}{stoichiometrisch}{stoichiometric}
\nomenclaturesubscript{t}{totaal}{total}
\nomenclaturesubscript{th}{theoretisch}{theoretical}
\nomenclaturesubscript{tot}{totaal}{total}
\nomenclaturesubscript{u}{universeel}{universal}
\nomenclaturesubscript{univ}{universum}{universe}
\nomenclaturesubscript{\specvol}{bij constant volume}{at constant volume}
\end{supertabular}

\mode<all> 
% Required because of the \mode* command at the beginning of this file.
% See beameruserfile.pdf section 21.3 for more info

